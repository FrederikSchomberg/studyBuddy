\documentclass[12pt,a4paper]{article}
\usepackage[ngerman]{babel}
\usepackage{geometry}
\usepackage{titlesec}

\geometry{a4paper, margin=2.5cm}
\titleformat{\section}{\normalfont\Large\bfseries}{\thesection}{1em}{}

\title{%
    Projektdokumentation Webtech2\\
    \large 4.Semester\\
    \vspace{0.5cm}
    \textbf{Studdy Buddy}
}

\author{
    Patrick Worm\\
    Jan Kaltofen\\
    Frederik Schomberg\\
    \\
    Hochschule Bochum-Informatik\\
}
\date{}

\begin{document}

\maketitle

\tableofcontents
\newpage

\section{Einleitung}
Jeder Student in der Welt soll einfach und bequem die Möglichkeit haben eine Lerngruppe zu finden, organisieren und sich auszutauschen. Egal ob vor Ort oder online!

\section{Vision}
Studdy Buddy unterstützt Lernende dabei, ihre Lernzeit zu planen, fokussiert zu bleiben und gemeinsam mit anderen Lernfortschritte zu erzielen. Alles an einem Ort.

\section{Epic}
\subsection{Lernorganisation optimieren}
\textbf{Beschreibung:}\\
Dieser Epic umfasst alle Funktionen, die Lernende beim strukturieren, Planen und Umsetzen ihrer Lernziele unterstützen.Individuell und in Gruppen.

\section{Personas}

\begin{itemize}
  \item Vollzeitstudent: Ich bin viel an der Hochschule und möchte einen online Treffpunkt für meine Gruppe finden.
  \item Arbeitnehmer während des Studiums: Ich habe wenig zeit und will Leute finden die an den selben tage zu den selben Zeiten bereit sind zu lernen.
  \item Student aus einer anderen Stadt: Ich kenne in dieser Stadt niemand und muss Leute finden mit den ich lernen kann.
\end{itemize}

\section{User Stories}
\subsection{Feature 1: Lerngruppen erstellen und verwalten}
\textbf{User Story:}\\
Als Vollzeitstudent möchte ich eine Lerngruppe erstellen können, damit ich mit anderen Studenten zusammenarbeiten kann.\\\\
\textbf{Feature:}
\begin{enumerate}
    \item Gruppenlogik (Erstellen und Beitreten über code) in Backend und UI umsetzen.
    \item Gruppenchat und gemeinsame Lernziele einführen.
\end{enumerate}

\subsection{Feature 2: Zeitplaner für Lernphasen}

\textbf{User Story:}\\
Als Nutzer*in möchte ich meinen Lernplan flexibel erstellen und anpassen können, damit ich gezielt auf Prüfungen oder Abgaben hinarbeiten kann.\\[1em]

\textbf{Feature:}
\begin{enumerate}
  \item UI zur Erstellung und Verwaltung eines Lernkalenders mit drag-and-drop-Funktion.
  \item Erinnerungsfunktion und Synchronisierung mit Kalenderdiensten.
  \item Backend-Speicherung pro Nutzer mit Anpassbarkeit nach Kursen oder Themen.
\end{enumerate}

\subsection{Feature 3: Fokustimer mit Lernstatistik}

\textbf{User Story:}\\
Als Nutzer*in möchte ich einen Fokustimer nutzen können, um konzentriert in Intervallen zu lernen und meinen Fortschritt nachvollziehen zu können.\\[1em]

\textbf{Feature:}
\begin{enumerate}
  \item Pomodoro-Timer mit konfigurierbaren Intervallen und automatischen Pausen.
  \item Statistikseite mit Lernzeit pro Tag/Woche sowie Diagrammen.
  \item Integration mit Benutzerprofil und Zielsetzung.
\end{enumerate}

\subsection{Feature 4: Lernziele setzen und verfolgen}

\textbf{User Story:}\\
Als Nutzer*in möchte ich Lernziele setzen und meinen Fortschritt sehen können, damit ich motiviert bleibe und mein Ziel im Blick habe.\\[1em]

\textbf{Feature:}
\begin{enumerate}
  \item UI zum Anlegen, Bearbeiten und Abhaken von Lernzielen.
  \item Visualisierung des Fortschritts mit Fortschrittsbalken.
  \item Option zur Verknüpfung mit Lerngruppen (gemeinsame Ziele).
\end{enumerate}

\section{Funktionale Anforderungen}
\begin{itemize}
  \item Lerngruppen erstellen und verwalten.
  \item Zeitplaner
  \item Lernziele setzen
  \item  Fokustimer
  \item registrieren und einloggen.
  \item Inhalte erstellen und bearbeiten.
\end{itemize}

\section{Technologien}
\begin{itemize}
  \item Frontend: HTML, CSS, JavaScript
  \item Backend: 
  \item Datenbank: 
\end{itemize}

\end{document}